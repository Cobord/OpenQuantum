\documentclass[11pt]{article}
\usepackage[margin=1in]{geometry} 
\geometry{letterpaper}   

\usepackage{amsmath}
\usepackage{amssymb,amsfonts,bbm,mathrsfs,stmaryrd}
\usepackage{url}

%%% Theorems and references %%%
\usepackage[amsmath,thmmarks]{ntheorem}
\usepackage{hyperref}
\usepackage{cleveref}

\theoremstyle{change}

\newtheorem{defn}[equation]{Definition}
\newtheorem{definition}[equation]{Definition}
\newtheorem{thm}[equation]{Theorem}
\newtheorem{theorem}[equation]{Theorem}
\newtheorem{prop}[equation]{Proposition}
\newtheorem{proposition}[equation]{Proposition}
\newtheorem{lemma}[equation]{Lemma}
\newtheorem{cor}[equation]{Corollary}
\newtheorem{conj}[equation]{Conjecture}
\newtheorem{conjecture}[equation]{Conjecture}
\newtheorem{exercise}[equation]{Exercise}
\newtheorem{example}[equation]{Example}

\theorembodyfont{\upshape}
\theoremsymbol{\ensuremath{\Diamond}}
\newtheorem{eg}[equation]{Example}
\newtheorem{remark}[equation]{Remark}

\theoremstyle{nonumberplain}

\theoremsymbol{\ensuremath{\Box}}
\newtheorem{proof}{Proof}

\qedsymbol{\ensuremath{\Box}}

\creflabelformat{equation}{#2(#1)#3} 

\crefname{equation}{equation}{equations}
\crefname{eg}{example}{examples}
\crefname{defn}{definition}{definitions}
\crefname{prop}{proposition}{propositions}
\crefname{thm}{Theorem}{Theorems}
\crefname{lemma}{lemma}{lemmas}
\crefname{cor}{corollary}{corollaries}
\crefname{remark}{remark}{remarks}
\crefname{section}{Section}{Sections}
\crefname{subsection}{Section}{Sections}

\crefformat{equation}{#2equation~(#1)#3} 
\crefformat{eg}{#2example~#1#3} 
\crefformat{defn}{#2definition~#1#3} 
\crefformat{prop}{#2proposition~#1#3} 
\crefformat{thm}{#2Theorem~#1#3} 
\crefformat{lemma}{#2lemma~#1#3} 
\crefformat{cor}{#2corollary~#1#3} 
\crefformat{remark}{#2remark~#1#3} 
\crefformat{section}{#2Section~#1#3} 
\crefformat{subsection}{#2Section~#1#3} 

\Crefformat{equation}{#2Equation~(#1)#3} 
\Crefformat{eg}{#2Example~#1#3} 
\Crefformat{defn}{#2Definition~#1#3} 
\Crefformat{prop}{#2Proposition~#1#3} 
\Crefformat{thm}{#2Theorem~#1#3} 
\Crefformat{lemma}{#2Lemma~#1#3} 
\Crefformat{cor}{#2Corollary~#1#3} 
\Crefformat{remark}{#2Remark~#1#3} 
\Crefformat{section}{#2Section~#1#3} 
\Crefformat{subsection}{#2Section~#1#3} 


\numberwithin{equation}{section}


%%% Tikz stuff %%%

\usepackage{tikz}
\tikzset{dot/.style={circle,draw,fill,inner sep=1pt}}
\usepackage{braids}
\usepackage{tqft}
\usetikzlibrary{tqft}
\usetikzlibrary{cd}
\usetikzlibrary{arrows}
%%% Letters, Symbols, Words %%%

\newcommand\Aa{{\cal A}}
\newcommand\Oo{{\cal O}}
\newcommand\Uu{{\cal U}}
\newcommand\NN{{\mathbb N}}
\newcommand\RR{{\mathbb R}}
\newcommand\Ddd{\mathscr{D}}
\renewcommand{\d}{{\,\rm d}}
\newcommand\T{{\rm T}}

\newcommand\mono{\hookrightarrow}
\newcommand\sminus{\smallsetminus}
\newcommand\st{{\textrm{ s.t.\ }}}
\newcommand\ket[1]{\mid #1 \rangle}
\newcommand\bra[1]{\langle #1 \mid}
\newcommand\setof[1]{\{ #1 \}}
\newcommand\lt{<}
\newcommand\abs[1]{ \mid #1 \mid }
\newcommand\conjbar[1]{\overline{#1}}

\DeclareMathOperator{\Aut}{Aut}
\DeclareMathOperator{\dVol}{dVol}
\DeclareMathOperator{\ev}{ev}
\DeclareMathOperator{\fiber}{fiber}
\DeclareMathOperator{\GL}{GL}
\DeclareMathOperator{\id}{id}
\DeclareMathOperator{\sign}{sign}
\DeclareMathOperator{\tr}{tr}


\title{Lindblad}
\author{Ammar Husain}

\begin{document}
\maketitle

\section{Lie Geometry of Density Matrix}

Rather than parameterize $\rho$ through matrix entries, parameterize it via an exponential family. Use the known importance of generalized Gibbs ensembles. Also reveals the Poisson geometry underlying Fisher metric.

\section{Lindblad}

\begin{eqnarray*}
\dot{\rho} (t) &=& - i [ H , \rho (t) ] + \sum_{k=1}^M L_k \rho (t) L_k^\dagger - \frac{1}{2} \setof{ L_k^\dagger L_k , \rho (t) }\\
\end{eqnarray*}

The action of $H$ on $V \otimes V^\dagger$ is given by coproduct and a sign flip for the $V^\dagger$ (More generally the antipode, but in this case just $-1$).

\begin{eqnarray*}
\Delta H &=& H \otimes 1 - 1 \otimes H\\
\end{eqnarray*}

\begin{eqnarray*}
\dot{\rho} &\in& V \otimes V^\dagger\\
\dot{\rho} &=& (-i \Delta H + \mathcal{G} ) \rho\\
\mathcal{G} &=& \sum_{m=0}^M \bar{L}_m \otimes L_m - \frac{1}{2} Id \otimes L_m^\dagger L_m - \frac{1}{2} \bar{L}_m^\dagger \bar{L}_m \otimes Id\\
\end{eqnarray*}

\section{Effective Hamiltonian}

\begin{definition}[$H_{eff}$]
\begin{eqnarray*}
H_{eff} &=& H - i \sum_k \frac{\gamma_k}{2} L_k^\dagger L_k\\
\end{eqnarray*}

The antiHermitian part is negative for $\gamma_k \geq 0$ in the sense of $L_k^\dagger L_k$ being a positive operator.

In Lie algebraic terms $i H_{eff} \in \mathfrak{u}(n) \bigoplus \mathfrak{u}(n)^*_{\geq 0}$

\begin{definition}[Iwasawa Poisson-Lie Group]
The one coming from $KAN$ decomposition. So for $GL(n,\mathbb{C})$ regarded as a real Lie group, we get Hermitian and antiHermitian split. TODO: Write down this parameterization.
\end{definition}

\begin{eqnarray*}
\frac{d\rho}{dt} &=& - i ( H_{eff} \rho - \rho H_{eff}^\dagger ) + \sum_{k} \gamma_k L_k \rho L_k^\dagger
\end{eqnarray*}

\section{Lie Parameterization}

Let $\rho = $

\end{definition}

\section{Examples}

\begin{example}
\begin{eqnarray*}
H &=& 0\\
\frac{\partial \rho}{\partial t} &=& -i [0,\rho] + \frac{\gamma}{2} ( \sigma_z \rho \sigma_z - \frac{1}{2} \setof{ \sigma_z^\dagger \sigma_z , \rho } )\\
\end{eqnarray*}
\end{example}

\begin{example}
\begin{eqnarray*}
H &=& \hbar \omega a^\dagger a\\
L_1 &=& a\\
\gamma_1 &=& \frac{\gamma}{2} (\bar{n}+1)\\
L_2 &=& a^\dagger\\
\gamma_2 &=& \frac{\gamma}{2} (\bar{n})\\
H_{eff} &=& \hbar \omega a^\dagger a - i \frac{\gamma}{4} ( (\bar{n}+1) a^\dagger a + (\bar{n}) a a^\dagger )\\
&=& \hbar \omega a^\dagger a - i \frac{\gamma}{4} ( (2 \bar{n}+1) a^\dagger a + \bar{n} )\\
\end{eqnarray*}
\end{example}

\section{Hochschild Homology}

\url{https://books.google.com/books?id=_9UBCAAAQBAJ&pg=PA88&lpg=PA88&dq=lindblad+hochschild&source=bl&ots=HI7IaXfgEe&sig=kOu64rDabBZnV024_gnlgv0pcl4&hl=en&sa=X&ved=0ahUKEwictbSF7MTXAhUKjlQKHfiRBWUQ6AEILTAB#v=onepage&q=lindblad\%20hochschild&f=false}

\end{document}