\documentclass[11pt]{article}
\usepackage[margin=1in]{geometry} 
\geometry{letterpaper}   

\usepackage{amsmath}
\usepackage{amssymb,amsfonts,bbm,mathrsfs,stmaryrd}
\usepackage{float}

%%% Theorems and references %%%
\usepackage[amsmath,thmmarks]{ntheorem}
\usepackage{hyperref}
\usepackage{cleveref}

\theoremstyle{change}

\newtheorem{defn}[equation]{Definition}
\newtheorem{definition}[equation]{Definition}
\newtheorem{thm}[equation]{Theorem}
\newtheorem{theorem}[equation]{Theorem}
\newtheorem{prop}[equation]{Proposition}
\newtheorem{proposition}[equation]{Proposition}
\newtheorem{lemma}[equation]{Lemma}
\newtheorem{cor}[equation]{Corollary}
\newtheorem{conjecture}[equation]{Conjecture}
\newtheorem{exercise}[equation]{Exercise}
\newtheorem{example}[equation]{Example}

\theorembodyfont{\upshape}
\theoremsymbol{\ensuremath{\Diamond}}
\newtheorem{eg}[equation]{Example}
\newtheorem{remark}[equation]{Remark}

\theoremstyle{nonumberplain}

\theoremsymbol{\ensuremath{\Box}}
\newtheorem{proof}{Proof}

\qedsymbol{\ensuremath{\Box}}

\creflabelformat{equation}{#2(#1)#3} 

\crefname{equation}{equation}{equations}
\crefname{eg}{example}{examples}
\crefname{defn}{definition}{definitions}
\crefname{prop}{proposition}{propositions}
\crefname{thm}{Theorem}{Theorems}
\crefname{lemma}{lemma}{lemmas}
\crefname{cor}{corollary}{corollaries}
\crefname{remark}{remark}{remarks}
\crefname{section}{Section}{Sections}
\crefname{subsection}{Section}{Sections}

\crefformat{equation}{#2equation~(#1)#3} 
\crefformat{eg}{#2example~#1#3} 
\crefformat{defn}{#2definition~#1#3} 
\crefformat{prop}{#2proposition~#1#3} 
\crefformat{thm}{#2Theorem~#1#3} 
\crefformat{lemma}{#2lemma~#1#3} 
\crefformat{cor}{#2corollary~#1#3} 
\crefformat{remark}{#2remark~#1#3} 
\crefformat{section}{#2Section~#1#3} 
\crefformat{subsection}{#2Section~#1#3} 

\Crefformat{equation}{#2Equation~(#1)#3} 
\Crefformat{eg}{#2Example~#1#3} 
\Crefformat{defn}{#2Definition~#1#3} 
\Crefformat{prop}{#2Proposition~#1#3} 
\Crefformat{thm}{#2Theorem~#1#3} 
\Crefformat{lemma}{#2Lemma~#1#3} 
\Crefformat{cor}{#2Corollary~#1#3} 
\Crefformat{remark}{#2Remark~#1#3} 
\Crefformat{section}{#2Section~#1#3} 
\Crefformat{subsection}{#2Section~#1#3} 


\numberwithin{equation}{section}


%%% Tikz stuff %%%

\usepackage{rotating}
\usepackage{tikz}
\tikzset{dot/.style={circle,draw,fill,inner sep=1pt}}
\usepackage{braids}
\usetikzlibrary{chains,matrix,scopes,decorations,shapes,arrows}
\usetikzlibrary{cd}

\newcommand{\hexagon}[2]{
\fill [blue] (-1/2+#1, .866+#2) circle (0.1);
\fill [blue] (1/2+#1, .866+#2) circle (0.1);
\fill [blue] (1+#1, 0+#2) circle (0.1);
\fill [blue] (-1+#1, 0+#2) circle (0.1);
\fill [blue] (-1/2+#1, -.866+#2) circle (0.1);
\fill [blue] (1/2+#1, -.866+#2) circle (0.1);
}

%%% Letters, Symbols, Words %%%

\newcommand\Aa{{\cal A}}
\newcommand\Oo{{\cal O}}
\newcommand\Uu{{\cal U}}
\newcommand\NN{{\mathbb N}}
\newcommand\RR{{\mathbb R}}
\newcommand\Ddd{\mathscr{D}}
\renewcommand{\d}{{\,\rm d}}
\newcommand\T{{\rm T}}

\newcommand\mono{\hookrightarrow}
\newcommand\sminus{\smallsetminus}
\newcommand\st{{\textrm{ s.t.\ }}}
\newcommand\ket[1]{\mid #1 \rangle}
\newcommand\bra[1]{\langle #1 \mid}
\newcommand\setof[1]{\{ #1 \}}
\newcommand\lt{<}
\newcommand\abs[1]{ \mid #1 \mid }
\newcommand\pfrac[2]{ \frac{ \partial #1}{\partial #2}}

\DeclareMathOperator{\Aut}{Aut}
\DeclareMathOperator{\dVol}{dVol}
\DeclareMathOperator{\ev}{ev}
\DeclareMathOperator{\fiber}{fiber}
\DeclareMathOperator{\GL}{GL}
\DeclareMathOperator{\id}{id}
\DeclareMathOperator{\sign}{sign}
\DeclareMathOperator{\tr}{tr}
\DeclareMathOperator{\Tr}{tr}

\title{Gate Identification}
\author{Ammar Husain}

\begin{document}
\maketitle

\section{Josephson Oscillator}

%%\url{https://copilot.caltech.edu/documents/261-girvin_notes_superconducting_qubits_and_circuits_2011.pdf}

Recall a single qubit constructed from a Josephson junction. The qubits are the $\ket{0}$ and $\ket{1}$ states of the number operator. The gates are then constructed by adjusting $E_J$ and $E_C$.

\begin{eqnarray*}
H_T &=& \frac{-1}{2} E_J \sum_m \ket{m} \bra{m+1} + \ket{m+1}\bra{m}\\
\ket{\phi} &=& \sum e^{im\phi} \ket{m}\\
H_T \ket{\phi} &=& -E_J \cos \phi \ket{\phi}\\
v_{group} ( \phi ) &=& \frac{1}{\hbar} \frac{\partial}{\partial \phi} ( - E_J \cos \phi )\\
I ( \phi ) &=& \frac{2e}{\hbar} E_J \sin \phi\\
\end{eqnarray*}

This gives a maximal coherent (dissipationless) current $\frac{2e}{\hbar} E_J$. Any more current will cause the voltage to rise above the $2\Delta$ gap and break the approximation that we only need the ground states determined by the number of pairs only.

Now with a Coulomb term as well for this little capacitor

\begin{eqnarray*}
H &=& 4 E_C ( n - n_g)^2 - E_J \cos \phi\\
\end{eqnarray*}

\begin{definition}[Mathieu Equation]
\begin{eqnarray*}
\big[ \frac{d^2}{dx^2} + ( a - 2 q \cos (2x) ) \big] y &=& 0\\
\end{eqnarray*}

For fixed $a$ and $q$, the solution can be expressed as $F(a,q,x)=e^{i \mu x} P(a,q,x)$ with $P$ $\pi$-periodic. 

\end{definition}

\begin{definition}[Hill Operator]

More generally, let $q(x)$ be a periodic real function instead of just the $a - 2 q_0 \cos 2x - \lambda$ as above.

\begin{eqnarray*}
H y &=& - y'' + q(x) y\\
H y &=& \lambda y\\
\end{eqnarray*}

\end{definition}

\begin{definition}[Mathieu cosine and sine]
The solution with value $1$ at $x=0$ and derivative 0 or vice versa.

These are real valued solutions.

\begin{eqnarray*}
C(a,q,x) &=& \frac{F(a,q,x)+F(a,q,-x)}{2 F(a,q,0)}\\
S(a,q,x) &=& \frac{F(a,q,x)-F(a,q,-x)}{2 F^{'} (a,q,0) }\\
\end{eqnarray*}

If $q=0$ then these are $\cos \sqrt{a} x$ and $\frac{\sin \sqrt{a} x}{\sqrt{a}}$ respectively. In general they are aperiodic.

For a given value of $q$ there are countably many $a$ that give periodic solutions. For example, if $q=0$, then $a = (n)^2$ with $n \in \mathbb{Z}$ in order to achieve $2 \pi$ periodicity.

\end{definition}

\begin{eqnarray*}
H &=& -4 E_C \frac{\partial^2}{\partial \phi^2} + 8E_C n_g i \frac{\partial}{\partial \phi} + 4E_C n_g^2 - E_J \cos \phi\\
H_2 &=& \frac{-H}{4 E_C} + n_g^2 = \frac{\partial^2}{\partial \phi^2} - 2 n_g i \frac{\partial}{\partial \phi} + \frac{E_J}{4E_C}\cos \phi\\
H_2 \psi = \lambda_{H_2} \psi  &\implies& -4 E_C (\lambda_{H_2} -n_g^2) = \lambda_{H}\\
\end{eqnarray*}

Let $\phi = 2x$

\begin{eqnarray*}
\frac{1}{4} \frac{\partial^2 \psi }{\partial x^2} - n_g i \frac{\partial \psi}{\partial x} + \frac{E_J}{4E_C}\cos (2x) \psi &=& \lambda \psi\\
\frac{\partial^2 \psi }{\partial x^2} - 4 n_g i \frac{\partial \psi}{\partial x} + (-4\lambda+\frac{E_J}{E_C}\cos (2x)) \psi &=& 0\\
\end{eqnarray*}

If $n_g=0$ then we get $\psi = a_S C(-4 \lambda , \frac{-E_J}{2E_C} , \frac{\phi}{2} ) + a_A S(-4 \lambda , \frac{-E_J}{2E_C} , \frac{\phi}{2} )$

Now keep $n_g$, but define $w$

\begin{eqnarray*}
w &=& \psi e^{-2 i n_g x} \\
w &=& C_1 C( - 4 \lambda + 4 n_g^2 , - \frac{E_J}{2E_C} , x ) + C_2 S( - 4 \lambda + 4 n_g^2 , - \frac{E_J}{2E_C} , x ) \\
\psi &=& e^{2 i n_g x} w\\
\end{eqnarray*}

\begin{proof}
\begin{eqnarray*}
\frac{\partial \psi}{\partial x} &=& \frac{\partial w}{\partial x} e^{2 i n_g x} + (2 i n_g ) w e^{2 i n_g x}\\
\frac{\partial^2 \psi}{\partial x^2} &=& \frac{\partial^2 w}{\partial x^2} e^{2 i n_g x} + 2i n_g \frac{\partial w}{\partial x} e^{2 i n_g x} + 2 i n_g \frac{\partial w}{\partial x} e^{2 i n_g x} - 4 n_g^2 w e^{2 i n_g x}\\ 
\frac{\partial^2 \psi}{\partial x^2} - 4 n_g i \frac{\partial \psi}{\partial x} + (-4 \lambda + \frac{E_J}{E_C} \cos 2x ) \psi &=& \frac{\partial^2 w}{\partial x^2} e^{2 in_g x} + 4 n_g^2 w e^{2 i n_g x} - ( 4 \lambda + \frac{E_J}{E_C} \cos 2x ) w e^{2i n_g x}\\
\frac{\partial^2 w}{\partial x^2} + (-4 \lambda + 4 n_g^2 - 2 \frac{E_J}{2E_C} \cos 2x ) w &=& 0\\
w &=& C_1 C(-4 \lambda + 4 n_g^2 , \frac{E_J}{2 E_C} , x ) + C_2 S(-4 \lambda + 4 n_g^2 , \frac{E_J}{2 E_C} , x )\\
\psi &=& C_1 e^{i n_g \phi} C(-4 \lambda + 4 n_g^2 , \frac{E_J}{2 E_C} , \frac{\phi}{2} ) + C_2 e^{ i n_g \phi} S(-4 \lambda + 4 n_g^2 , \frac{E_J}{2 E_C} , \frac{\phi}{2} )\\
\end{eqnarray*}

\end{proof}

We want $\psi$ to be periodic under $\phi \to \phi + 2 \pi$ so $x \to x+ \pi$. So $w$ must go to $e^{-2 i n_g \pi}w $ upon the same shift. This gives countably many allowed values for $-4 \lambda + 4 n_g^2$ so countably many values for the energy $\lambda_H$. That is the spectrum as a function of $E_{C,J}$ and $n_g$.

Let's say $C_1=1$ and $C_2 = 0$. In Mathematica this is given by MathieuCharacteristicA[$-2 n_g+k$,$-E_J/2E_C$]. Call it $MCA$ for short. That is under $x \to x + 2 \pi$ pick up a $e^{i r 2 \pi}$.

\begin{eqnarray*}
\lambda &=& n_g^2 -\frac{1}{4} MCA ( k - 2 n_g, -\frac{E_J}{2E_C} )\\
\end{eqnarray*}

Similarly for the odd functions we have 

\begin{eqnarray*}
\lambda &=& n_g^2 -\frac{1}{4} MCB ( k - 2 n_g, -\frac{E_J}{2E_C} )\\
\end{eqnarray*}

Together the spectrum is the collection of all

\begin{eqnarray*}
\lambda_H &=& E_C MCA ( k -2  n_g, -\frac{E_J}{2E_C} )\\
\lambda_H &=&  E_C MCB ( k -2  n_g, -\frac{E_J}{2E_C} )\\
\end{eqnarray*}

\section{More general 1 qubit}

Notice that if we define $y = e^{i \phi}$, $\cos \phi$ has degrees $\pm 1$ in $y$. In addition the Coulomb term is quadratic in $n=i \hbar \frac{d}{d\phi}$. Therefore, for unknown dynamics $H$ let us guess a form

\begin{eqnarray*}
H &=& A_{0,0} + A_{0,1} n + A_{1,0} e^{i \phi} + A_{-1,0} e^{- i \phi} \\
&+& A_{0,2} n^2 + A_{1,1} e^{i \phi} n + A_{-1,1} e^{-i\phi} n \\
&+& A_{2,0} e^{2 i \phi} + A_{-2,0} e^{-2 i \phi}\\
&+& h.c.
\end{eqnarray*}

which has all the $A_{m,n}$ such that $\abs{m} + n \leq 2$. Generally let the maximal degree be $D$.

\section{Multiple qubits}

Let $n_1 \cdots n_k$ be the number operators and $\phi_i \cdots \phi_k$ corresponding phases. Again maximal degree $D$.

\begin{eqnarray*}
H &=& \sum A_{a_1,b_1,a_2,b_2 \cdots a_k , b_k} \prod_i e^{i a_i \phi_i} n_i^{b_i} + h.c.\\
\sum_i \abs{a}_i + b_i &\leq& D\\
\end{eqnarray*}

The total number of parameters here is upper bounded by $2^{k} \sum_{n \leq D} p(n,2k)$ where the $2^k$ takes care of choosing the signs on the $a_i$ and the $p(n)$ are the number of compositions of $D$ into $2k$ natural numbers. If $D$ is too big, this is unmanageable, but for smaller $D$ it is more reasonable.

See HamiltonianLearning repository for the corresponding classical problem with the same kind of constraint being used to make the inverse dynamic problem solvable.

\begin{eqnarray*}
\langle H \rangle &=& \sum \langle A_{a_1,b_1,a_2,b_2 \cdots a_k , b_k} \prod_i e^{i a_i \phi_i} n_i^{b_i} + h.c. \rangle\\
&=& \sum Re(A_{a_1,b_1,a_2,b_2 \cdots a_k , b_k}) \langle \prod_i e^{i a_i \phi_i} n_i^{b_i} + h.c. \rangle\\
&+& \sum Im(A_{a_1,b_1,a_2,b_2 \cdots a_k , b_k}) \langle i( \prod_i e^{i a_i \phi_i} n_i^{b_i} - h.c.) \rangle\\
\end{eqnarray*}

If we have the statistics on those expectation values, then we get a linear algebra problem over the reals for the unknown $Re(A)$ and $Im(A)$. WLOG we have demanded that $\langle H \rangle = 0$ by adjusting $A_{0 \cdots 0}$. In order to get these statistics we must do repeated observations where the state was initialized the same and then allowed to evolve for some time. For each of these $\rho(t=0)$ and $t$ you get one row of the matrix. Do this with many different initial conditions and evolution times and you get more rows. We get the problem of finding the null space of a rectangular matrix. Because of errors, instead of taking the null space just do an SVD and impose a cutoff on the absolute value of singular values that should be considered. That provides a fuzzy version of the null space to take into account potential errors. 

\subsection{Bootstrapping}

We first want a good guess for the Hamiltonian. This is done by identifying the case from the more general 1 qubit. Call that $H_{i}$ for it only applied to qubit $i$. The initial guess for $k$ qubits is then

\begin{eqnarray*}
H &=& \sum_{j=1}^k H_i
\end{eqnarray*}

$H_i$ have only $A_{0 \cdots 0, a_i , b_i, 0 \cdots 0}$ nonzero. Then do similar for $H_{ij}$ to focus on the case of $2$-qubits in isolation from the other $k-2$. $2$-local Hamiltonians are typical so one might stop there. These give smaller matrices for which we seek the null space. However, because of the approximation used again just use the SVD decomposition and take the vectors correpsonding to the smallest singular values.

\end{document}